%%%%%%%%%% !TEX program = pdflatex %%%%%%%%%%
% !TEX program = pdflatex
% !Mode:: "TeX:UTF-8"
% \def\usewhat{dvipdfmx}                              % 定义编译方式 dvipdfmx 或者 pdflatex ,默认为 dvipdfmx
													% 方式编译,如果需要修改,只需改变花括号中的内容即可。
%\setlength{\baselineskip}{20pt}
%\setlength{\headheight}{25pt}
\documentclass[a4paper,12.5pt,openany,twoside]{book}

                                           % 如果论文超过60页 可以使用twoside 双面打印
\input{setup/package}                      % 定义本文所使用宏包
\graphicspath{{figures/}}                  % 定义所有的.eps文件在figures子目录下
\begin{document}                           % 开始全文
\begin{CJK*}{UTF8}{song}                   % 开始中文字体使用
\input{setup/format}                       % 完成对论文各个部分格式的设置
\frontmatter                               % 以下是论文导言部分,包括论文的封面,中英文摘要和中文目录
% !Mode:: "TeX:UTF-8"

\chnunumer{10532}
\chnuname{湖南大学}
\cclassnumber{TP391}
\cnumber{S191801513}
\csecret{普通}
\cmajor{金融工程学}
\cheading{硕士学位论文}      % 设置正文的页眉,以及自己的学位级别
\dtitle{高维因子模型的极大似然分析的理论与方法}%封面用论文标题,已经手动断行
\ctitle{高维因子模型的极大似然分析的理论与方法}  %页眉标题无需断行
\etitle{High-dimension Factor Model}
\caffil{金融与统计学院} %学院名称
\csubjecttitle{学科专业}
\csubject{金融学}   %专业
\cauthortitle{研究生}     % 学位
\cauthor{张三}   %学生姓名
\ename{ZHANG~~San}
\cbe{B.A.~(Shanghai University of Finance and Economics)2018}
\cms{M.A.~(Hunan University)2022}
\cdegree{dissertation}
\cclass{Master of Art}
\emajor{Finance}
\ehnu{Hunan~University}
\esupervisor{LI, Si}
\csupervisortitle{指导教师}
\elevel{Professor}% 导师职称
\csupervisor{李四~~教授} %导师姓名
\cchair{ }
\ddate{2022年4月30日}
\edate{April,~2022}

\untitle{湖~~南~~大~~学}
\declaretitle{学位论文原创性声明}
\declarecontent{
本人郑重声明:所呈交的论文是本人在导师的指导下独立进行研究所取得的研究成果。除了文中特别加以标注引用的内容外,本论文不包含任何其他个人或集体已经发表或撰写的成果作品。对本文的研究做出重要贡献的个人和集体,均已在文中以明确方式标明。本人完全意识到本声明的法律后果由本人承担。
}
\authorizationtitle{学位论文版权使用授权书}
\authorizationcontent{
本学位论文作者完全了解学校有关保留、使用学位论文的规定,同意学校保留并向国家有关部门或机构送交论文的复印件和电子版,允许论文被查阅和借阅。本人授权湖南大学可以将本学位论文的全部或部分内容编入有关数据库进行检索,可以采用影印、缩印或扫描等复制手段保存和汇编本学位论文。
}
\authorizationadd{本学位论文属于}
\authorsigncap{作者签名:}
\supervisorsigncap{导师签名:}
\signdatecap{签字日期:}


%\cdate{\CJKdigits{\the\year} 年\CJKnumber{\the\month} 月 \CJKnumber{\the\day} 日}
% 如需改成二零一二年四月二十五日的格式,可以直接输入,即如下所示
% \cdate{二零一二年四月二十五日}
\cdate{\the\year 年\the\month 月 \the\day 日} % 此日期显示格式为阿拉伯数字 如2012年4月25日
\cabstract{
在过去的20年里,经济学见证了经济统计工作的飞跃发展。
}

\ckeywords{高维分位数因子模型}

\eabstract{
In order to investigate the dynamic connectedness among the four variables...

}

\ekeywords{High-dimension factor model}

\makecover

\clearpage
                      % 封面
%%%%%%%%%%   目录   %%%%%%%%%%
\defaultfont
\addcontentsline{toc}{chapter}{目~~~~录}
\tableofcontents                           % 中文目录
\clearpage
\newcommand{\loflabel}{图~}
\renewcommand{\numberline}[1]{\song\xiaosi\loflabel~#1\hspace*{\baselineskip}}
\addcontentsline{toc}{chapter}{插图索引}
\listoffigures
\clearpage
\newcommand{\lotlabel}{表~}
\renewcommand{\numberline}[1]{\song\xiaosi\lotlabel~#1\hspace*{\baselineskip}}
\addcontentsline{toc}{chapter}{附表索引}
\listoftables
\clearpage{\pagestyle{empty}\cleardoublepage}
%%%%%%%%%% 正文部分内容  %%%%%%%%%%
\mainmatter\defaultfont\sloppy\raggedbottom
\renewcommand{\ALC@linenosize}{\xiaosi}
\renewcommand\arraystretch{1.5}
\newcommand\Mycite[1]{\citeauthor{#1}~,\citeyear{#1}}
\newcommand{\cy}[1]{(\citeyear{#1})}
\newcommand{\mc}[1]{\citeauthor{#1}(\citeyear{#1})}
\newcommand{\mcc}[1]{\citeauthor*{#1}(\citeyear{#1})}
\newcommand{\cov}{\mathrm{Cov}}
\newcommand{\var}{\mathrm{Var}}
\newcommand{\n}{\mathrm{N}}

%%% New command setting
\newtheorem{thm}{定理}[section]
\newtheorem{lem}{引理}[section]
\newtheorem{alg}{算法}[section]
\newtheorem{prop}{性质}[section]
\numberwithin{equation}{section}
%\renewcommand{\theequation}{\thechapter.\arabic{equation}}
\renewcommand{\proofname}{\bf 证明}
\renewcommand{\qedsymbol}{$\blacksquare$}
\newcommand{\sixvdots}{%
	\vbox{\baselineskip1ex\lineskiplimit0pt%
		\hbox{.}\hbox{.}\hbox{.}\hbox{.}\hbox{.}\hbox{.}}}
\newcommand{\rmd}{\, \mathrm{d}}
\newcommand{\bbK}{\mathbb{K}}
\newcommand{\Data}{\mathrm{Data}}
\newcommand{\din}{\mathrm{in}}
\newcommand{\dover}{\mathrm{over}}
\newcommand{\calL}{\mathcal{L}}
\newcommand{\calK}{\mathcal{K}}
\newcommand{\doublestar}{{\ast\ast}}
\newcommand{\triplestar}{{\ast\ast\ast}}
\makeatletter
\newcommand{\rmnum}[1]{\romannumeral #1}
\newcommand{\Rmnum}[1]{\expandafter\@slowromancap\romannumeral #1@}
\makeatother
\setlength{\intextsep}{2pt}
\setlength{\abovecaptionskip}{2pt}
\setlength{\belowcaptionskip}{2pt}
% !Mode:: "TeX:UTF-8"

\chapter{绪论}
\section{研究背景}

\clearpage

% !Mode:: "TeX:UTF-8"

\chapter{高维因子模型的极大似然分析的理论与方法}


% !Mode:: "TeX:UTF-8"

\chapter{参数估计}\label{para_est}

% !Mode:: "TeX:UTF-8"

\chapter{变量选取和指数构造} \label{chap:vars_index}




% !Mode:: "TeX:UTF-8"

\chapter{实证结果}



\include{body/chap6}
%%%%%%%%%% 正文部分内容  %%%%%%%%%%

%%%%%%%%%%  参考文献  %%%%%%%%%%
\defaultfont
\bibliographystyle{gbt7714-numerical}
\phantomsection
\addcontentsline{toc}{chapter}{参考文献}          % 参考文献加入到中文目录
\nocite{*}                                        % 若将此命令屏蔽掉,则未引用的文献不会出现在文后的参考文献中。
\bibliography{ref.bib}
% !Mode:: "TeX:UTF-8"
\addcontentsline{toc}{chapter}{附录A~~~~抽样算法}
\chapter*{附录A~~~~抽样算法}
\renewcommand{\theequation}{A.\arabic{equation}}
\newtheorem{\thelem}{\textbf{引理}}
\renewcommand{\thelem}{A.\arabic{lem}}
\renewcommand{\thetable}{A.\arabic{table}}
\setlength{\parindent}{2em}
\setcounter{lem}{0}
\setcounter{equation}{0}
\setcounter{table}{0}

%\gdef\thepage{A\arabic{page}}
\addcontentsline{toc}{section}{附录A.1~~算法细节}
\section*{A.1~~算法细节}

 \begin{lem} \label{lemma1}
     令$N_{t+1}=\mathrm{Var}(y_{t+1}\vert y^{0,t},x_t,\bbK^{1,t+1})$。于是,
     \begin{align}
         N_{t+1}&=h_{t+1}^\prime \tilde{\Gamma}_{t+1}\tilde{\Gamma}_{t+1}^\prime h_{t+1}+G_{t+1}G_{t+1}^\prime \\
         \mathbb{E}(y_{t+1}\vert y^{0,t},x_t,\bbK^{1,t+1})&=g_{t+1}+h_{t+1}^\prime (f_{t+1}+F_{t+1}x_t)
     \end{align} 
     $(x_{t+1}\vert y^{0,t+1},x_t,\bbK)$的均值和方差为:
     \begin{align}
         \mathbb{E}(x_{t+1}\vert y^{0,t+1},x_t,\bbK)&=a_{t+1}+A_{t+1}x_t+B_{t+1}y_{t+1} \\
         \mathrm{Var}(x_{t+1}\vert y^{0,t+1},x_t,\bbK)&=C_{t+1}C_{t+1}^\prime
     \end{align}
     其中,
     \begin{align}
        a_{t+1}&=(I-B_{t+1}h_{t+1}^\prime)f_{t+1}-B_{t+1}g_{t+1}\\
        A_{t+1}&=(I-B_{t+1}h_{t+1}^\prime)F_{t+1} \\
        B_{t+1}&=\tilde{\Gamma}_{t+1}\tilde{\Gamma}_{t+1}^\prime h_{t+1}N_{t+1}^{-1} \\
        C_{t+1}C_{t+1}^\prime&=\tilde{\Gamma}_{t+1}\tilde{\Gamma}_{t+1}^\prime-B_{t+1}N_{t+1}B_{t+1}^\prime
     \end{align}
     假设,
     \begin{equation}
        x_{t+1}=a_{t+1}+A_{t+1}x_t+B_{t+1}y_{t+1}+C_{t+1}z_{t+1}
     \end{equation}
     其中,$z_{t+1}\vert\bbK\sim \mathcal{N}(0,I)$独立于$x_t$和$y_{t+1}$。
 \end{lem}
 \begin{proof}
     \begin{align*}
        \mathrm{Cov}(x_{t+1},y_{t+1})&=\mathbb{E}\left[(x_{t+1}-\mathbb{E}(x_{t+1}\vert \bbK))(y_{t+1}-\mathbb{E}(y_{t+1}\vert \bbK))^\prime\right] \\
        &=\mathbb{E}\left\{[B_{t+1}(y_{t+1}-\mu_{y_{t+1}})+C_{t+1}z_{t+1}](y_{t+1}-\mu_{y_{t+1}})^\prime\right\} \\
        &=B_{t+1}N_{t+1}
     \end{align*}
     \begin{align*}
        \mathrm{Cov}(x_{t+1},y_{t+1})&=\mathrm{Cov}(x_{t+1},\ g_{t+1}+h_{t+1}^\prime x_{t+1}+G_{t+1}u_{t+1}) \\
         &= \mathbb{E}\left[(x_{t+1}-\mu_{x_{t+1}})((x_{t+1}-\mu_{x_{t+1}})^\prime h_{t+1}+u_{t+1}^\prime G_{t+1}^\prime)\right] \\
         &= \tilde{\Gamma}_{t+1}\tilde{\Gamma}_{t+1}^\prime h_{t+1}
     \end{align*}
     \begin{equation}
         B_{t+1}=\tilde{\Gamma}_{t+1}\tilde{\Gamma}_{t+1}^\prime h_{t+1} N_{t+1}^{-1}
     \end{equation} 
     显然,
     \begin{equation}
        \begin{split}
            C_{t+1}C_{t+1}^\prime &= \mathrm{Var}_{t+1}(x\vert y)=\tilde{\Gamma}_{t+1}\tilde{\Gamma}_{t+1}^\prime
                    -B_{t+1}N_{t+1}N_{t+1}^{-1}N_{t+1}B_{t+1}^\prime \\
                &= \tilde{\Gamma}_{t+1}\tilde{\Gamma}_{t+1}^\prime -B_{t+1}N_{t+1}B_{t+1}^\prime
        \end{split}
     \end{equation}
     \begin{equation*}
         \begin{split}
             x_{t+1} &= a_{t+1}+A_{t+1}x_t+B_{t+1}(g_{t+1}+h_{t+1}^\prime x_{t+1}+G_{t+1}\mu_{t+1}) \\
             &= a_{t+1}+A_{t+1}x_t+B_{t+1}g_{t+1}+B_{t+1}h_{t+1}^\prime f_{t+1}+B_{t+1}h_{t+1}^\prime x_t+
                B_{t+1}h_{t+1}^\prime \tilde{\Gamma}_{t+1}v_{t+1}+B_{t+1}G_{t+1}\mu_{t+1} \\
             &=f_{t+1}+F_{t+1}x_t+\tilde{\Gamma}_{t+1}v_{t+1} \\
         \end{split}
     \end{equation*}
     因此,
     \begin{equation*}
         \begin{split}
             A_{t+1}+B_{t+1}h_{t+1}^\prime F_{t+1} &= F_{t+1} \\
             a_{t+1} + B_{t+1}g_{t+1}+B_{t+1}h_{t+1}^\prime f_{t+1} &= f_{t+1}
         \end{split}
     \end{equation*}
\end{proof}


\begin{table}[!htbp]
    \centering
    \caption{混合分布中对应的七个正态分布的参数}
    \begin{threeparttable}
        \begin{tabular}{cccc}
            \hline
            $\omega$ & $q_j=\mathrm{Pr}\{\omega=j\}$ & $m_j$ & $v_j^2$ \\
            \hline
            1 & 0.00730 & $-$10.12999 & 5.79596 \\
            2 & 0.10556 & $-$3.97281  & 2.61369 \\
            3 & 0.00002 & $-$8.56686  & 5.17950 \\
            4 & 0.04395 & 2.77786   & 0.16735 \\
            5 & 0.34001 & 0.61942   & 0.64009 \\
            6 & 0.24566 & 1.79518   & 0.34023 \\
            7 & 0.25750 & $-$1.08819  & 1.26261 \\
            \hline
        \end{tabular}
        \begin{tablenotes}
            \footnotesize
            \item[*] 来源: \mc{1998Kim}
        \end{tablenotes}
    \end{threeparttable}
\label{tab:log-chi-sq-1}
\end{table}

\clearpage
\addcontentsline{toc}{section}{附录A.2~~后验分布}
\section*{附录A.2~~后验分布}
本节将介绍部分参数的后验分布。


\addcontentsline{toc}{chapter}{附录B~~~~发表论文和参加科研情况说明}
\chapter*{附录B~~~~发表论文和参加科研情况说明}
\setlength{\parindent}{0em}
\textbf{(一)发表的学术论文}
\begin{publist}
\item 
\end{publist}

\vspace*{1em}
\textbf{(二)申请及已获得的专利(无专利时此项不必列出)}
\begin{publist}
\item XXX,XXX. XXXXXXXXX:中国,1234567.8[P]. 2012-04-25.
\end{publist}
\vspace*{1em}
\textbf{(三)参与的科研项目}
\begin{publist}
\item	XXX,XXX. XX~信息管理与信息系统, ~国家自然科学基金项目.课题编号:XXXX.
\end{publist}
\vfill
\hangafter=1\hangindent=2em\noindent

\setlength{\parindent}{2em}
                   % 发表论文和参加科研情况说明
% !Mode:: "TeX:UTF-8"
\addcontentsline{toc}{chapter}{致\quad 谢} %添加到目录中
\chapter*{致\quad 谢}




               % 致谢
\clearpage
\end{CJK*}                                        % 结束中文字体使用
\end{document}                                    % 结束全文
